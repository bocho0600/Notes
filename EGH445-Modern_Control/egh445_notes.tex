\documentclass{article}
\usepackage{amsmath}
\usepackage{amssymb}
\usepackage{physics}
\usepackage{fancyhdr}
\usepackage[a4paper, margin=0.8in]{geometry}

\pagestyle{fancy}
\fancyhf{}
\lhead{EGH445 - Modern Control}
\rfoot{Page \thepage}

\begin{document}

\title{Summary of Equations in Aerodynamics Lecture Notes}
\author{By Kelvin Le}
\date{Modified: \today}
\maketitle

\tableofcontents

\section{Introduction}
EGH445 is a course that explores fundamental engineering principles, modeling
techniques, system dynamics, and control strategies. The course integrates
theoretical and practical knowledge, focusing on modern computational tools.

\section{System Modelling}
System modeling involves representing a physical system mathematically to
analyze and predict its behavior. Two major approaches include:
\begin{itemize}
      \item \textbf{Input-State-Output Model:}
            \begin{align*}
                  \dot{x} & = f(x, u) \\
                  y       & = g(x, u)
            \end{align*}
            where:
            \begin{itemize}
                  \item $x$ is the state vector
                  \item $u$ is the input vector
                  \item $y$ is the output vector
            \end{itemize}

      \item \textbf{Linear Time-Invariant (LTI) System:}
            \begin{align*}
                  \dot{x} & = Ax + Bu \\
                  y       & = Cx + Du
            \end{align*}
            where $A$, $B$, $C$, and $D$ are system matrices.
\end{itemize}

\subsection{State-Space Representation}
The state-space model describes the system using state variables:
\begin{align*}
      x & = \begin{bmatrix} x_1 \\ x_2 \\ \vdots \\ x_n \end{bmatrix},
      \quad u = \begin{bmatrix} u_1 \\ u_2 \\ \vdots \\ u_m \end{bmatrix},
      \quad y = \begin{bmatrix} y_1 \\ y_2 \\ \vdots \\ y_p \end{bmatrix}
\end{align*}
This representation allows a complete description of system dynamics.

\subsection{State-Space Model of an Ordinary Differential Equation (ODE)}

We consider a second-order differential equation of the form:

\begin{align*}
      \ddot{y}(t) + \mathbf{a} \dot{y}(t) + \mathbf{b} y(t) &= u(t)
\end{align*}

Rearranging for \( \ddot{y}(t) \):

\begin{align*}
      \ddot{y}(t) &= -\mathbf{a} \dot{y}(t) - \mathbf{b} y(t) + u(t)
\end{align*}

\subsubsection{Step 1: Define the State Variables}

Define the state variables as:

\begin{align*}
      x_1(t) &= y(t), \\
      x_2(t) &= \dot{y}(t)
\end{align*}

Substituting into the equation, we obtain the state-space representation:

\begin{align*}
      \dot{x}_1(t) &= x_2(t), \\
      \dot{x}_2(t) &= -\mathbf{a} x_2(t) - \mathbf{b} x_1(t) + u(t)
\end{align*}

\subsubsection{Step 2: Express in Matrix Form}
The state-space equations can be rewritten in matrix form as:

\begin{align*}
    \begin{bmatrix} \dot{x}_1 \\ \dot{x}_2 \end{bmatrix} &= 
    \begin{bmatrix} 0 & 1 \\ -\mathbf{b} & -\mathbf{a} \end{bmatrix} 
    \begin{bmatrix} x_1 \\ x_2 \end{bmatrix} + 
    \begin{bmatrix} 0 \\ 1 \end{bmatrix} u(t)
\end{align*}

\subsubsection{Step 3: Define the Output Equation}
If we consider the output \( y(t) \) as one of the state variables, the output equation is:

\begin{align*}
    y(t) &= \begin{bmatrix} 1 & 0 \end{bmatrix} 
    \begin{bmatrix} x_1 \\ x_2 \end{bmatrix}
\end{align*}

\subsubsection{Step 4: Final State-Space Representation}
Combining the state and output equations, the complete state-space representation is:

\begin{align*}
    \dot{x}(t) &= A x(t) + B u(t), \\
    y(t) &= C x(t) + D u(t)
\end{align*}

where:

\begin{align*}
    A &= \begin{bmatrix} 0 & 1 \\ -\mathbf{b} & -\mathbf{a} \end{bmatrix}, \quad
    B = \begin{bmatrix} 0 \\ 1 \end{bmatrix}, \quad
    C = \begin{bmatrix} 1 & 0 \end{bmatrix}, \quad
    D = \begin{bmatrix} 0 \end{bmatrix}
\end{align*}

This represents the final state-space form of the given second-order ODE.

\subsection{Transfer Function of a State-Space Model}
\begin{align*}
      \begin{bmatrix} \dot{x} \\ \dot{x} \end{bmatrix}, \quad = \begin{bmatrix} 0 & 1 \\ -\mathbf{b} & -\mathbf{a} \end{bmatrix}, \quad
  \end{align*}




\section{Control System Analysis}
A control system regulates the behavior of dynamic systems. Key elements
include:
\begin{itemize}
      \item \textbf{Open-loop Control:} Control action does not depend on output.
      \item \textbf{Closed-loop Control:} Control action adjusts based on feedback.
\end{itemize}

\subsection{Transfer Function Representation}
The transfer function relates input and output in the frequency domain:
\begin{align*}
      G(s) = \frac{Y(s)}{U(s)} = \frac{b_0 + b_1s + \dots + b_m s^m}{a_0 + a_1s + \dots + a_n s^n}
\end{align*}
where $s$ is the Laplace variable.

\section{Stability Analysis}
Stability determines if a system returns to equilibrium after disturbances.
\begin{itemize}
      \item A system is \textbf{stable} if all poles of its transfer function have negative
            real parts.
      \item A system is \textbf{unstable} if any pole has a positive real part.
      \item \textbf{Marginal Stability:} Poles are on the imaginary axis.
\end{itemize}

\section{Feedback Control Systems}
Feedback control systems use sensors to adjust control inputs dynamically.
\begin{itemize}
      \item \textbf{Proportional Control (P):} $u(t) = K_p e(t)$
      \item \textbf{Proportional-Integral Control (PI):} $u(t) = K_p e(t) + K_i \int e(t) dt$
      \item \textbf{Proportional-Integral-Derivative (PID):}
            \begin{align*}
                  u(t) = K_p e(t) + K_i \int e(t) dt + K_d \frac{de(t)}{dt}
            \end{align*}
\end{itemize}
where $e(t)$ is the error signal, and $K_p$, $K_i$, $K_d$ are control gains.

\section{Frequency Response Analysis}
The frequency response method analyzes system behavior under sinusoidal inputs.
\begin{itemize}
      \item \textbf{Bode Plot:} Logarithmic magnitude and phase plots.
      \item \textbf{Nyquist Criterion:} Determines stability using contour mapping.
      \item \textbf{Gain and Phase Margins:} Measures robustness.
\end{itemize}

\section{Conclusion}
EGH445 covers key principles of system modeling, stability, and control,
emphasizing theoretical analysis and computational tools.

\end{document}
