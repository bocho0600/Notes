\documentclass{article}
\usepackage{amsmath}
\usepackage{amssymb}
\usepackage{physics}
\usepackage{fancyhdr}
\usepackage[a4paper, margin=0.8in]{geometry}

\pagestyle{fancy}
\fancyhf{}
\lhead{Aerodynamics Equations Summary}
\rfoot{Page \thepage}

\begin{document} 

\title{Summary of Equations in Aerodynamics Lecture Notes}
\author{By Kelvin Le}
\date{Modified: \today}
\maketitle

\tableofcontents

\section{Fundamental Aerodynamics Equations}

\subsection{Equation of State for a Gas (Ideal Gas Law)}
\begin{equation}
    p = \rho R T
\end{equation}
where:
\begin{itemize}
    \item $p$ = Pressure (Pa or N/m$^2$)
    \item $\rho$ = Density (kg/m$^3$)
    \item $R$ = Specific Gas Constant (J/kg$\cdot$K)
    \item $T$ = Temperature (K)
\end{itemize}

\subsection{Specific Volume}
\begin{equation}
    v = \frac{1}{\rho}
\end{equation}
where $v$ is the specific volume (m$^3$/kg).

\subsection{Kinetic Energy and Temperature Relation}
\begin{equation}
    K_e = \frac{3}{2} k T
\end{equation}
where:
\begin{itemize}
    \item $K_e$ = Average kinetic energy of molecules (J)
    \item $k$ = Boltzmann constant ($1.38 \times 10^{-23}$ J/K)
    \item $T$ = Temperature (K)
\end{itemize}

\subsection{Shear Stress Due to Viscosity (Newton’s Law of Friction)}
\begin{equation}
    \tau = \mu \frac{du}{dy}
\end{equation}
where:
\begin{itemize}
    \item $\tau$ = Shear stress (Pa or N/m$^2$)
    \item $\mu$ = Dynamic viscosity (Pa$\cdot$s or N$\cdot$s/m$^2$)
    \item $\frac{du}{dy}$ = Velocity gradient (s$^{-1}$)
\end{itemize}

\subsection{Reynolds Number (Flow Characterization)}
\begin{equation}
    Re = \frac{\rho V l}{\mu} = \frac{V l}{\nu}
\end{equation}
where:
\begin{itemize}
    \item $Re$ = Reynolds number (dimensionless)
    \item $V$ = Flow velocity (m/s)
    \item $l$ = Characteristic length (m)
    \item $\mu$ = Dynamic viscosity (Pa$\cdot$s or N$\cdot$s/m$^2$)
    \item $\nu$ = Kinematic viscosity (m$^2$/s)
\end{itemize}

\subsection{Mach Number (Compressibility Effects)}
\begin{equation}
    M = \frac{V}{a}
\end{equation}
where:
\begin{itemize}
    \item $M$ = Mach number (dimensionless)
    \item $V$ = Object speed (m/s)
    \item $a$ = Speed of sound in the medium (m/s)
\end{itemize}

\section{Example Calculations}

\subsection{Finding Temperature Using Ideal Gas Law}
\begin{equation}
    T = \frac{p}{\rho R}
\end{equation}
Example:
\begin{equation}
    T = \frac{8.9876 \times 10^4}{(1.1117)(287)} = 281K
\end{equation}

\subsection{Air Weight in a Room}
First, use the equation of state to find $\rho$:
\begin{equation}
    \rho = \frac{p}{RT}
\end{equation}
Then, find mass:
\begin{equation}
    m = \rho V
\end{equation}
where $m$ is mass (kg) and $V$ is volume (m$^3$).

Finally, weight:
\begin{equation}
    W = mg
\end{equation}
where $g$ is gravitational acceleration (9.81 m/s$^2$), and $W$ is weight (N).

\subsection{Percentage Change in Air Weight Due to Temperature Drop}
Since the temperature changes, we recalculate density:
\begin{equation}
    \rho = \frac{2116}{(1716)(460 -10)} = 0.00274 \frac{\text{slug}}{\text{ft}^3}
\end{equation}
Compare densities:
\begin{equation}
    \Delta \rho = 0.00274 - 0.00237 = 0.00037 \frac{\text{slug}}{\text{ft}^3}
\end{equation}
Percentage change:
\begin{equation}
    \% \text{change} = \frac{\Delta \rho}{\rho} \times 100 = \frac{0.00037}{0.00237} \times 100 = 15.6\%\text{ increase}
\end{equation}

Alternative solution:
\begin{equation}
    \Delta W = \Delta m \cdot g = 0.888 \times 32.2 = 28.5936 \text{ lb}
\end{equation}
\begin{equation}
    \% \text{change} = \frac{\Delta W}{W_1} \times 100 = \frac{28.5936}{183} \times 100 = 15.6\%
\end{equation}

\end{document}
